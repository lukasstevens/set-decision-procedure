\documentclass[sigplan,10pt,anonymous,review]{acmart}
\settopmatter{printfolios=true,printccs=false,printacmref=false}

\usepackage[british]{babel}

\usepackage{mathpartir}

%\usepackage{unicode-math}
%\setmathfont{latinmodern-math.otf}
%\setmathfont[version=bold, FakeBold=2]{latinmodern-math.otf}

\usepackage{xcolor}

\usepackage{academicons}

\usepackage{tikz}
\usepackage{pgfplots}
\pgfplotsset{compat=1.16}

\usepackage{enumitem}

\setlength{\abovecaptionskip}{0pt}
\setlength{\textfloatsep}{18pt plus 2pt minus 4pt}

\usepackage{listings}
\definecolor{isarblue}{HTML}{006699}
\definecolor{isargreen}{HTML}{009966}
\lstdefinelanguage{isabelle}{%
    keywords=[1]{type_synonym,datatype,fun,abbreviation,definition,proof,lemma,theorem,corollary},
    keywordstyle=[1]\bfseries\color{isarblue},
    keywords=[2]{where,assumes,shows,and},
    keywordstyle=[2]\bfseries\color{isargreen},
    keywords=[3]{if,then,else,case,of,SOME,let,in,O},
    keywordstyle=[3]\color{isarblue},
}
\lstset{%
  language=isabelle,
  escapeinside={&}{&},
  columns=fixed,
  extendedchars,
  basewidth={0.5em,0.45em},
  basicstyle=\ttfamily,
  mathescape,
}
\makeatletter
\lst@AddToHook{OnEmptyLine}{\vspace{-0.4\baselineskip}}
\makeatother

\usepackage{calc}

\newcommand{\textover}[3][l]{%
 % #1 is the alignment, default l
 % #2 is the text to be printed
 % #3 is the text for setting the width
 \makebox[\widthof{#3}][#1]{#2}%
}

\hyphenation{Isa-belle}

\pagestyle{plain} % turn on page numbers in llncs

% % LuaLatex on Arxiv: https://tex.stackexchange.com/questions/372154/lualatex-how-to-produce-pdf-acceptable-by-arxiv 
% \hypersetup{%
%   pdfcreator = {},
%   pdfproducer = {}
% }
% \pdfvariable suppressoptionalinfo \numexpr 1+2+4+8+16+32+64+128+256+512 \relax

\begin{document}

\setcounter{tocdepth}{1}

\title{Verifying a Tableau Prover for Multi-Level-Syllogistics}
\author{Lukas Stevens}
\orcid{0000-0003-0222-6858}
\affiliation{%
  \institution{Technical University of Munich}
  \department{Department of Informatics}
  \streetaddress{Boltzmannstr. 3}
  \city{Garching}
  \postcode{85748}
  \country{Germany}
}
\email{lukas.stevens@in.tum.de}

\begin{abstract}
  TODO
\end{abstract}

\maketitle

\section{Introduction}
TODO

\subsection{Related work}
TODO

\subsection{Contributions}
TODO

\subsection{Notation}
Isabelle/HOL~\cite{isabelle} conforms to everyday mathematical notation for the most part.
For the benefit of the reader that is unfamiliar with Isabelle/HOL, we establish notation and in particular some essential datatypes together with their primitive operations that are specific to Isabelle/HOL.
We write \lstinline!t :: 'a! to specify that the term \lstinline!t! has the type \lstinline!'a! and \lstinline!'a $\Rightarrow$ 'b! for the type of a total function from \lstinline!'a! to \lstinline!'b!.
The types for booleans, natural numbers, and in integers are \lstinline!bool!, \lstinline!nat!, and \lstinline!int!, respectively.
Sets with elements of type \lstinline!'a! have the type \lstinline!'a set!.
Analogously, we use \lstinline!'a list! to describe lists, which are constructed as the empty list \lstinline![]! or with the infix constructor \lstinline!#!, and are appended with the infix operator \lstinline!@!.
The function \lstinline!set! converts a list into a set.
For optional values, Isabelle/HOL offers the type \lstinline!option! where a term \lstinline!opt :: 'a option! is either \lstinline!None! or \lstinline!Some a! with \lstinline!a :: 'a!.
Finally, we remark that \textbf{iff} is equivalent to \lstinline!$=$! on type \lstinline!bool! and \lstinline!$\equiv$! is definitional equality of the meta-logic of Isabelle/HOL, which is called Isabelle/Pure.

\section{A Semantics for Sets\label{sec:semantics}}
TODO

\section{Conclusion}
TODO

\begin{acks}
  TODO
\end{acks}

\bibliographystyle{ACM-Reference-Format}
\bibliography{sources}

\end{document}
